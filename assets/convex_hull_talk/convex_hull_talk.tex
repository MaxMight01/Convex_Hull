\documentclass[12pt]{article}
\usepackage{amsmath, amssymb, graphicx, hyperref, enumerate}
\usepackage{xcolor, algorithm, algpseudocode, subcaption, float}
\usepackage{mathrsfs}

\title{Convex Hull Algorithms: Jarvis’s March and Graham’s Scan}
\author{Ramdas Singh}
\date{}

\newcommand{\R}{\mathbb{R}}
\newcommand{\N}{\mathbb{N}}
\newcommand{\C}{\mathcal{C}}
\newcommand{\CH}{\mathcal{CH}}

\begin{document}
\maketitle

\section*{Convexity}
\textit{Introduce the rubberband.}\\

A subset $\C \subseteq \R^{2}$ is called \textit{convex} if and only if for all $p,q \in S$, $(1-\lambda) p + \lambda q \in \C$ for all $\lambda \in [0,1]$. Other equivalent definitions---
\begin{itemize}
    \item A set $\C \subseteq \R^{2}$ is convex if and only if it can be expressed as the intersection of (possible infinitely many) closed half-spaces.
    \item A set $\C \subseteq \R^{2}$ is convex if and only if for every point $p \in \R^{2}\backslash \C$, there exists a hyper plane diving the space into one open half-space containing $p$ and one closed half-space containing $\C$.
\end{itemize}
Definition can be easily extended to higher dimensions.

\section*{Convex Hull}
The \textit{convex hull} $\CH(S)$ of a set $S \subseteq \R^{2}$ is the `smallest' convex set that contains $S$. More precisely,
\begin{itemize}
    \item The convex hull $\CH(S)$ of a set $S$ is the intersection of all convex sets that contain $S$:
    \[
        \CH(S) = \bigcap_{\lambda \in \Lambda} \{C \subseteq \R^{n} : S \subseteq C,\; C \text{ is convex}\}.
    \]
    \item The convex hull $\CH(S)$ of a set $S$ consists of all convex combinations of finitely many points in $S$:
    \[
        \CH(S) = \{\sum_{i=1}^{k} \lambda_{i}p_{i} : p_{i} \in S,\; \lambda_{i} \geq 0,\; \sum_{i=1}^{k} \lambda_{i} = 1,\; k \in \N\}
    \]
\end{itemize}

\section*{Vertices}
The \textit{vertices}, intuitively, are simply the vertices of the polygon formed by the convex hull. Formally, the \textit{non-vertices} are defined as
\[
    \overline{V}(S) = \{p \in S: \CH(S) = \CH(S\backslash\{p\})\}
\]
Thus, the vertices become
\[
    V(S) = S\backslash \overline{V}(S).
\]

\subsection*{Problem Statement}
Given a set $S = \{p_{1},p_{2},\ldots,p_{n}\}$ of points in $\R^{2}$, compute the ordered set $V(S)$, ordered in clockwise direction.


\section*{Interesting (and Helpful) Results}
First introduce Jarvis and Graham; they developed fantastic convex hull algorithms for this problem statement.

\subsection*{Caratheodory's Theorem (1907)}
If $p \in \CH(S)$ for $S \subseteq \R^{2}$, then $p$ can be expressed as $p = \sum_{i=1}^{3} \lambda_{i} p_{i}$ where $\lambda_{i} \geq 0$ and $\sum_{i=1}^{k} \lambda_{i}$ for some $p_{i} \in S$.

All it says is, if $p$ is in the convex hull, then there exists a triangle containing $p$ whose vertices are formed by vertices in $S$. Gives us an idea that to determine if $p$ is a vertex or not, we need only consider triangles; gave insight to Jarvis that only small subset of poitns is needed to represent a convex hull, and that they can be computed incrementally.

\subsection*{Radon's Theorem (1921)}
Any set of 4 points in $\R^{2}$ can be partitioned into two sets whose convex hulls intersect. (Proof is a good exercise). Evoked the idea of divide and conquer.


\section*{Jarvis March (1973)}
\textit{R.~A.~Jarvis}. Involved in robotics, computer vision, path finding, and image processing.

\subsection*{Concept (Bruteforce wrapping)}
\begin{enumerate}
    \item Start with an empty set $V(S) = \emptyset$.
    \item Consider all $(p,q) \in S \times S$.
    \item If all $r \in S\setminus\{p,q\}$ exist to the right of the directed line $\ell(p,q)$, then set $V(S)$ to be $V(S) \cup \{p,q\}$.
    \item Order clockwise, and return.
\end{enumerate}

\newpage

\subsection*{Pseudocode}
\begin{algorithm}
\caption{Jarvis’s March (Bruteforce Wrapping) }
\begin{algorithmic}[1]
\Require A finite set of points \( S \subset \mathbb{R}^2 \).
\Ensure The set of extreme points \( V(S) \), arranged in a clockwise order.

\State Initialize \( V(S) \gets \emptyset \).
\ForAll{distinct pairs \( (p, q) \in S \times S \) with \( p \neq q \)}
    \State Assume \( \ell(p, q) \), the directed line through \( p \) and \( q \), is a boundary of \( V(S) \).
    \ForAll{points \( r \in S \setminus \{p, q\} \)}
        \If{\( r \) lies strictly to the left of \( \ell(p, q) \)}
            \State Discard \( \ell(p, q) \) as a boundary.
        \EndIf
    \EndFor
    \If{\( \ell(p, q) \) was never discarded}
        \State Include \( p \) and \( q \) in \( V(S) \).
    \EndIf
\EndFor

\State Order the elements of \( V(S) \) in a clockwise manner.
\State Return \( V(S) \).
\end{algorithmic}
\end{algorithm}


\subsection*{Time Complexity}
\[
O(n^{2}h) + O(h \log h) = O(n^{2}h), \quad h = \text{number of vertices}
\]
Really, on the worst case possible when every point in $S$ is a vertex, this is $O(n^3)$.

\textit{Revolutionary, but completely unoptimized.}\\
Gift wrapping version is much better; the concept goes as follows:

\subsection*{Concept (Gift wrapping)}
\begin{enumerate}
    \item Start with an empty set $V(S) = \emptyset$.
    \item Immediately add the `leftmost' point, $p_{0}$ to $V(S)$.
    \item You now only need to consider $(p_{0},p) \in S \times S$ where $p \neq p_{0}$.
    \item If all $r \in S\setminus\{p_{0},p\}$ exist to the right of the directed line $\ell(p,q)$, then set $p_{1} = p$ and add $p_{1}$ to $V(S)$.
    \item Continue with $(p_{1},p) \in S \times S$ where $p \neq p_{1}$ until you similarly find $p_{2}$. Keep continuing until $p_{h}$ is forced to be $p_{0}$. Return.
\end{enumerate}

\subsection*{Some Explanations}
\subsubsection*{What happens if $r$ lies on $\ell(p,q)$?}
We could simply add it, increasing the value of $h$ and increasing computation time. But a check for not adding it would preferable; this check would cost us $O(h)$, thus not increasing computation time by much.

\subsubsection*{How to check if $r$ lies to the right of $\ell(p,q)$?}
You could find the angle between $\ell(p,q)$ and $\ell(q,r)$. A better approach is to compute
\[
    D = \det \begin{pmatrix}
        1 & p_{x} & p_{y} \\
        1 & q_{x} & q_{y} \\
        1 & r_{x} & r_{y}
    \end{pmatrix}
\]
The sign of $D$ tells whether $r$ lies to the right (and on) or the left of the line.


\end{document}
